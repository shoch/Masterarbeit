% -------------------------------------------------------
% Daten für die Arbeit
% Wenn hier alles korrekt eingetragen wurde, wird das Titelblatt
% automatisch generiert. D.h. die Datei titelblatt.tex muss nicht mehr
% angepasst werden.

\newcommand{\hsmasprache}{de} % de oder en für Deutsch oder Englisch
% Für korrekt sortierte Literatureinträge, noch preambel.tex anpassen
% und zwar bei \usepackage[main=ngerman, english]{babel},
% \usepackage[pagebackref=false,german]{hyperref}
% und \usepackage[autostyle=true,german=quotes]{csquotes}

% Titel der Arbeit auf Deutsch
\newcommand{\hsmatitelde}{Automatisches Transferlernen mittels Autoencodern}

% Titel der Arbeit auf Englisch
\newcommand{\hsmatitelen}{Automatic transfer learning using autoencoders}

% Weitere Informationen zur Arbeit
\newcommand{\hsmaort}{Offenburg}    % Ort
\newcommand{\hsmaautorvname}{Sebastian} % Vorname(n)
\newcommand{\hsmaautornname}{Hoch} % Nachname(n)
\newcommand{\hsmadatum}{ XX.XX.2020} % Datum der Abgabe
\newcommand{\hsmajahr}{2020} % Jahr der Abgabe
\newcommand{\hsmafirma}{PSIORI GmbH} % Firma bei der die Arbeit durchgeführt wurde
\newcommand{\hsmabetreuer}{Prof. Dr.-Ing. Janis Keuper, Hochschule Offenburg} % Betreuer an der Hochschule
\newcommand{\hsmazweitkorrektor}{Dr. rer. nat. Sascha Lange, PSIORI GmbH} % Betreuer im Unternehmen oder Zweitkorrektor
\newcommand{\hsmafakultaet}{EMI} % Fakultät
\newcommand{\hsmastudiengang}{INFM} % Studiengangsabkürzung. 
% Diese wird in titelblatt.tex definiert. Bisher AI, EI, MK und INFM. Bitte ergänzen.

% Zustimmung zur Veröffentlichung
\setboolean{hsmapublizieren}{true}   
\setboolean{hsmasperrvermerk}{false} 

% -------------------------------------------------------
% Abstract

% Kurze (maximal halbseitige) Beschreibung, worum es in der Arbeit geht auf Deutsch
\newcommand{\hsmaabstractde}{Im Rahmen dieser Arbeit wurden drei Werkzeuge erstellt um Merkmalextraktion, Transferlernen und AutoMl zu kombinieren.
	Das erste Werkzeug gleicht einen Schwäche eines Autoencoders aus. Beim Training eines Autoencoders wird die Rekonstruktion, also der Output des Modelles und nicht direkt die Einbettung als Bewertungskriterium herangezogen. Um diese Schwäche zu kompensieren wurde der SCAE erstellt. Dieses Werkzeug ist ein Autoencoder mit weiterem Ausgang. Die Datenrepräsentation wird durch ein zweites Kriterium gestärkt. Das zweite Werkzeug nutzt die  Datenrepäsentation um einen Transferlearning Task durchzuführen. Das zweite Kriterium wird durch ein neues Kriterium ersetzt. Als drittes Werkzeug wurde der TCSCAE um funktionen des AutoMl erweitert. Die besten Hyperparameter werden automatisch gefunden. Die Werkzeuge wurden anhand von echten Datensätzen getestet und validiert. Dabei hat sich gezeigt, dass mit den Werkzeugen eine änhlich gute Leistung wie auf dem herkömmlichen Weg erreicht werden kann und das durch das Transferlern sogar aufwand reduziert werden kann.
	
	Beschreibugn der Tools weniger konkret?
}


% Kurze (maximal halbseitige) Beschreibung, worum es in der Arbeit geht auf Englisch

\newcommand{\hsmaabstracten}{Englische Version von Lorem ipsum dolor sit amet, consetetur sadipscing elitr, sed diam nonumy eirmod tempor invidunt ut labore et dolore magna aliquyam erat, sed diam voluptua. At vero eos et accusam et justo duo dolores et ea rebum. Stet clita kasd gubergren, no sea takimata sanctus est Lorem ipsum dolor sit amet. Lorem ipsum dolor sit amet, consetetur sadipscing elitr, sed diam nonumy eirmod tempor invidunt ut labore et dolore magna aliquyam erat, sed diam voluptua. At vero eos et accusam et justo duo dolores et ea rebum. Stet clita kasd gubergren, no sea takimata sanctus est Lorem ipsum dolor sit amet.}
