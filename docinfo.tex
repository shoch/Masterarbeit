% -------------------------------------------------------
% Daten für die Arbeit
% Wenn hier alles korrekt eingetragen wurde, wird das Titelblatt
% automatisch generiert. D.h. die Datei titelblatt.tex muss nicht mehr
% angepasst werden.

\newcommand{\hsmasprache}{de} % de oder en für Deutsch oder Englisch
% Für korrekt sortierte Literatureinträge, noch preambel.tex anpassen
% und zwar bei \usepackage[main=ngerman, english]{babel},
% \usepackage[pagebackref=false,german]{hyperref}
% und \usepackage[autostyle=true,german=quotes]{csquotes}

% Titel der Arbeit auf Deutsch
\newcommand{\hsmatitelde}{Aufgabenfokussierung auf Autoencoder und automatisches Transferlernen}

% Titel der Arbeit auf Englisch
\newcommand{\hsmatitelen}{Task focusing on autoencoder and automatic transfer learning}

% Weitere Informationen zur Arbeit
\newcommand{\hsmaort}{Offenburg}    % Ort
\newcommand{\hsmaautorvname}{Sebastian} % Vorname(n)
\newcommand{\hsmaautornname}{Hoch} % Nachname(n)
\newcommand{\hsmadatum}{30. Juni 2020} % Datum der Abgabe
\newcommand{\hsmajahr}{2020} % Jahr der Abgabe
\newcommand{\hsmafirma}{PSIORI GmbH} % Firma bei der die Arbeit durchgeführt wurde
\newcommand{\hsmabetreuer}{Prof. Dr.-Ing. Janis Keuper, Hochschule Offenburg} % Betreuer an der Hochschule
\newcommand{\hsmazweitkorrektor}{Dr. rer. nat. Sascha Lange, PSIORI GmbH} % Betreuer im Unternehmen oder Zweitkorrektor
\newcommand{\hsmafakultaet}{EMI} % Fakultät
\newcommand{\hsmastudiengang}{INFM} % Studiengangsabkürzung. 
% Diese wird in titelblatt.tex definiert. Bisher AI, EI, MK und INFM. Bitte ergänzen.

% Zustimmung zur Veröffentlichung
\setboolean{hsmapublizieren}{true}   
\setboolean{hsmasperrvermerk}{false} 

% -------------------------------------------------------
% Abstract

% Kurze (maximal halbseitige) Beschreibung, worum es in der Arbeit geht auf Deutsch
\newcommand{\hsmaabstractde}{
	Hohe Kosten bei der Annotation von Daten führen dazu, dass datensparsamere Wege zum Erstelllen von Modellen gesucht werden. In dieser Arbeit wird ein Lösungsansatz untersucht, der ausgehend von fokusierten Repäsentationen, datensparsame Lösungen für verschiedene Aufgaben finden soll. Durch einen Multi-Task-Ansatz trägt das Finden einer Reräsentation gleichzeitig zum Lösen einer Aufgabe bei. Durch Ersetzung einer der Multi-Task können Wissentransfere datensparsam auf neue Aufgabe durchgeführt werden. In der erarbeiteten und evaluierten Lösung können Parameter automatisch gefunden werden. Bei einem Vergleich von verschiedenen Ansätzen und einem Vergleich mit verschiedenen Datenmengen ist über die Leistung der Netzwerke zu erkennen, dass der Ansatz insbesondere mit weniger Daten bessere Ergebnisse erzielt. Die gute Leistung der Ansätze motiviert zu einer Bereitstellung als Module. Die Module werden im Rahmen dieser Arbeit beschrieben. Abgeschlossen wird die Arbeit mit einem Ausblick auf Verbesserungen und Potenziale der Ansätze.
}


% Kurze (maximal halbseitige) Beschreibung, worum es in der Arbeit geht auf Englisch

\newcommand{\hsmaabstracten}{
	\todo{Zusammenfasssung auf Englisch} Englische Version von Lorem ipsum dolor sit amet, consetetur sadipscing elitr, sed diam nonumy eirmod tempor invidunt ut labore et dolore magna aliquyam erat, sed diam voluptua. At vero eos et accusam et justo duo dolores et ea rebum. Stet clita kasd gubergren, no sea takimata sanctus est Lorem ipsum dolor sit amet. Lorem ipsum dolor sit amet, consetetur sadipscing elitr, sed diam nonumy eirmod tempor invidunt ut labore et dolore magna aliquyam erat, sed diam voluptua. At vero eos et accusam et justo duo dolores et ea rebum. Stet clita kasd gubergren, no sea takimata sanctus est Lorem ipsum dolor sit amet.
\todo{Abstract ins Englische übersetzen}}
