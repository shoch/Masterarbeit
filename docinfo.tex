% -------------------------------------------------------
% Daten für die Arbeit
% Wenn hier alles korrekt eingetragen wurde, wird das Titelblatt
% automatisch generiert. D.h. die Datei titelblatt.tex muss nicht mehr
% angepasst werden.

\newcommand{\hsmasprache}{de} % de oder en für Deutsch oder Englisch
% Für korrekt sortierte Literatureinträge, noch preambel.tex anpassen
% und zwar bei \usepackage[main=ngerman, english]{babel},
% \usepackage[pagebackref=false,german]{hyperref}
% und \usepackage[autostyle=true,german=quotes]{csquotes}

% Titel der Arbeit auf Deutsch
\newcommand{\hsmatitelde}{Aufgabenfokussierung auf Autoencoder und automatisches Transferlernen}

% Titel der Arbeit auf Englisch
\newcommand{\hsmatitelen}{Task focusing on autoencoder and automatic transfer learning}

% Weitere Informationen zur Arbeit
\newcommand{\hsmaort}{Waldkirch}    % Ort
\newcommand{\hsmaautorvname}{Sebastian} % Vorname(n)
\newcommand{\hsmaautornname}{Hoch} % Nachname(n)
\newcommand{\hsmadatum}{30. Juni 2020} % Datum der Abgabe
\newcommand{\hsmajahr}{2020} % Jahr der Abgabe
\newcommand{\hsmafirma}{PSIORI GmbH} % Firma bei der die Arbeit durchgeführt wurde
\newcommand{\hsmabetreuer}{Prof. Dr.-Ing. Janis Keuper, Hochschule Offenburg} % Betreuer an der Hochschule
\newcommand{\hsmazweitkorrektor}{Dr. rer. nat. Sascha Lange, PSIORI GmbH} % Betreuer im Unternehmen oder Zweitkorrektor
\newcommand{\hsmafakultaet}{EMI} % Fakultät
\newcommand{\hsmastudiengang}{INFM} % Studiengangsabkürzung. 
% Diese wird in titelblatt.tex definiert. Bisher AI, EI, MK und INFM. Bitte ergänzen.

% Zustimmung zur Veröffentlichung
\setboolean{hsmapublizieren}{true}   
\setboolean{hsmasperrvermerk}{false} 

% -------------------------------------------------------
% Abstract

% Kurze (maximal halbseitige) Beschreibung, worum es in der Arbeit geht auf Deutsch
\newcommand{\hsmaabstractde}{
	Hohe Kosten bei der Annotation von Daten führen dazu, dass datensparsamere Wege zum Erstelllen von Modellen gesucht werden. In dieser Arbeit wird ein Lösungsansatz untersucht, der ausgehend von fokusierten Repräsentationen, datensparsame Lösungen für verschiedene Aufgaben finden soll. Durch einen Multi-Task-Lernen-Ansatz trägt das Finden einer Repräsentation gleichzeitig zum Lösen einer Aufgabe bei. Durch Ersetzung einer der Aufgaben werden Wissentransfers datensparsam auf neue Aufgabe durchgeführt. In der erarbeiteten und evaluierten Lösung können Hyperparameter automatisch gefunden werden. Bei Vergleichen von verschiedenen Ansätzen und Datenmengen ist über die Leistung der Netzwerke zu erkennen, dass der Ansatz insbesondere mit weniger Daten bessere Ergebnisse erzielt. Die Ergebnisse dieser Arbeit lassen eine Bereitstellung als Module zu. Die Module werden im Rahmen dieser Arbeit beschrieben. Abgeschlossen wird die Arbeit mit einem Ausblick auf Verbesserungen und Potenziale der Ansätze.
}


% Kurze (maximal halbseitige) Beschreibung, worum es in der Arbeit geht auf Englisch

\newcommand{\hsmaabstracten}{
	High costs when annotating data leads to the search for ways to get along with a reduced amount of data when creating models. This work explores a possible approach to find solutions with less data, by making use of focused representations. A multi task learning approach allows found representations to simultaneously help finding solutions to a task. By replacing one of the tasks, a knowledge transfer to another task, which makes efficient use of data, can be made. In this developed and evaluated approach, parameters can be be found automatically. When making a comparison between different approaches and with different amounts of data, taking into account the networks score, it can be seen, that the new approach produces more optimal results – especially when working with a reduced amount of data. The high score of the approach was motivation to provide it as python modules. These modules will be described as part of this work. Finally a future prospect with possible vectors for improvement will be made.
}
