\chapter{Fazit}
\label{chap:Fazit}
Die Arbeit schließt mit einer Zusammenfassung der Inhalte ab, auf welche eine kritische Auseinandersetzung mit verschiedenen Ergebnissen der Abschlussarbeit folgt und durch einen ausführlichen Blick auf mögliche Erweiterungen und weitere geleistete Arbeiten komplettiert wird.
 
	\section{Zusammenfassung}
	\label{sec:Zusammenfassung}
	Zu Beginn der Arbeit wurde die zu bearbeitende Problemstellung erläutert, anschließend wurden die theoretischen Hintergründe, das Umfeld der Problemstellung und die Datensätze vorgestellt. Der Hauptteil der Arbeit beschäftigt sich mit aufeinander aufbauenden Ansätzen des Transferlernens. Im ersten Schritt wurde gezeigt, dass ein Ansatz, ausgehend von einer nicht fokussierten Repräsentation fehlschlägt. Darauf folgend wurde ein Multi-Task-Ansatz zum gleichzeitigen Fokussieren und Lösen einer Regressionsaufgabe vorgestellt und evaluiert. Insbesondere die gefundene Repräsentation hat dabei deutlich die aktuelle Domäne widergespiegelt. Darauf aufbauend wurde ein Ansatz des modellbasierten Transferlernen vorgestellt. Die Ergebnisse erreichen eine ähnliche Leistung wie ein Modell, welches von einem Experten erstellt wurde. Zur \ac{hpo} der Aufgaben-Gewichtungsparameter wurde eine \ac{automl}-Erweiterung eingesetzt. Dabei hat sich gezeigt, dass der Hyperparameter einen deutlichen Einfluss auf das Ergebnis hat und die automatische Suche hilfreich ist. Abschließend wurde gezeigt, dass die Transfer-Lösung insbesondere für geringe Datenmengen gute Ergebnisse liefern kann. Begleitend zu der Vorstellung der Ansätze wurde jeweils eine Python-Implementierung des Ansatzes vorgestellt. 
							
	\section{Ausblick und weitere Arbeiten}
	\label{sec:AusblickWeitereArbeiten}
	In diesem letzten Unterkapitel werden weitere (vorläufige) Ergebnisse dargestellt und ein Blick auf mögliche Erweiterungen geworfen.

	\subsection{Transfer auf Greiferdatensatz}
	\label{subsec:TransferGreiferDatensatz}
 	Die Annotationskosten sind je nach Art von Annotation und Aufwand unterschiedlich. Eine einfache Zuordnung eines Bildes zu einer, von zwei Klassen liegt dabei im einstelligen Cent-Bereich. Die Markierung eines Objektes in einem Bild mittels Rahmen verursacht Kosten im zweistelligen Cent-Bereich und die Markierung von Winkeln in einem Bild kann mehr als einen Euro kosten. In Experiment \ref{sec:TransferDatenmenge} und \ref{subsec:TransferLogs} wurde gezeigt, dass der Transfer für die Aufgabe 'Greifer beladen?' funktioniert und insbesondere bei wenigen Daten stark ist. Falls es möglich ist, dass ein Transfer von der Aufgabe 'Greifer beladen?' zu der Aufgabe 'Rahmen um Greifer' möglich ist, kann sehr viel Geld gespart werden, da die Annotationen deutlich teurer sind. 
 	
 	Im ersten Versuch ist der Transfer auf dem Greiferdatensatz fehlgeschalgen, durch weitere Arbeiten an dem Problem konnte eine Verbesserung erzielt werden. Zu den weiteren Arbeiten zählen, das Entfernen von Bildern bei denen der Greifer über den Rand des bildes hinausgeht. Diese Bidler können durch eine Modifikation der Kamerposition am Kran nicht mehr vorkommen. Weitere Verbesserungen konten durch Regularsierungstechniken erzielt werden. Im Anhang \ref{appendix:TTAEGreifer } ist die aktuelle Modellzusammenfassung zu finden.
 	In Abbildung \ref{img:TTGrappleIoU} sind die Ergebnisse der Aufgabe 'Rahmen um Greifer' zu sehen. Bei einem Schwellenwert von 0.5 wird eine Leistung von 96\% Erreicht, bei einem Schwellenwert 0.8 eine Leistung von 55\%. Bei genauerer Betrachtung der Vorhersagen fällt auf, dass  der Greifer gefunden wird. Die genaue Form des Greifers wird nicht imer korrekt Vorhergesagt. Dieses Verhalten ist auch in Abbildung \ref{img:TTGrappleLabelVsPred} zu sehen. In dieser Art von Abbildungen werden die vorhersagen gegen die Beschriftungen aufgetragen. Bei einer perfekten Vorhersage befinden sich alle Datenpunkte auf der Diagonale. Sowohl in der Darstellung der Rahmenhöhe als auch der Rahmenbreite , ist zu sehen, das die Vorhersage zur Beschriftung passt, aber sich noch verbessern kann.  In Abbildung \ref{img:TTGrappleEinbettungVorhersage} sind die für den Schwellen wert 0.8 falsch vorhergesagten Datenpunkte lila eingefärbt. Die falschen Vorhersagen verteilen sich über alle Datenpunkte. Mit Blick auf die Einfärbung der x und y Position des Greifers \ref{img:TTGrappleIoU} zeigt sich in den Einbettungen die Position des Greifers im Bild. Wiederspeigelnd zu den Vorhersageergebnissen ist in der Einbettung die genaue Ausprägung des Greifers schwach abbgebildet.
 	
 	 
 	\begin{figure}[h]
 		\centering
 		\begin{subfigure}[c]{0.32\textwidth}			
 			\includegraphics[width=1\textwidth]{bilder/FazitUndAusblick/Grapple_TTAE_Res/IoU_TTAE_05.png}
 			\subcaption{Schwellenwert 0.5}			
 		\end{subfigure}
 		\begin{subfigure}[c]{0.32\textwidth}			
 			\includegraphics[width=1\textwidth]{bilder/FazitUndAusblick/Grapple_TTAE_Res/IoU_TTAE_08.png}
 			\subcaption{Schwellenwert 0.8}			
 		\end{subfigure}
 		\begin{subfigure}[c]{0.32\textwidth}			
 			\includegraphics[width=1\textwidth]{bilder/FazitUndAusblick/Grapple_TTAE_Res/Recall_IoU_TTAE.png}
 			\subcaption{IoU-Recall-Curve}			
 		\end{subfigure}
 		\caption{IoU TTAE Greifer}
 		\label{img:TTGrappleIoU}
 	\end{figure}
 
 	 	\begin{figure}[h]
 		\centering
 		\begin{subfigure}[c]{0.49\textwidth}			
 			\includegraphics[width=1\textwidth]{bilder/FazitUndAusblick/Grapple_TTAE_Res/Height_BB.png}
 			\subcaption{Rahmenhöhe}			
 		\end{subfigure}
 		\begin{subfigure}[c]{0.49\textwidth}			
 			\includegraphics[width=1\textwidth]{bilder/FazitUndAusblick/Grapple_TTAE_Res/Width_BB.png}
 			\subcaption{Rahmenbreite}			
 		\end{subfigure}  
 		\caption{Beschriftung vs Vorhersage}
 		\label{img:TTGrappleLabelVsPred}
 	\end{figure}
 
 	\begin{figure}[h]
 		\centering
 		\begin{subfigure}[c]{0.32\textwidth}			
 			\includegraphics[width=1\textwidth]{bilder/FazitUndAusblick/Grapple_TTAE_Res/TT_Emb_y.png}
 			\subcaption{Einbettung markierte y-Position Greifer}			
 		\end{subfigure}
 		\begin{subfigure}[c]{0.32\textwidth}			
 			\includegraphics[width=1\textwidth]{bilder/FazitUndAusblick/Grapple_TTAE_Res/TT_Emb_x.png}
 			\subcaption{Einbettung markierte x-Position Greifer}			
 		\end{subfigure}
 		\begin{subfigure}[c]{0.32\textwidth}			
 			\includegraphics[width=1\textwidth]{bilder/FazitUndAusblick/Grapple_TTAE_Res/TT_Emb_pred.png}
 			\subcaption{Einbettung mit Vorhersage}		
 			\label{img:TTGrappleEinbettungVorhersage}	
 		\end{subfigure}
 		\caption{Einbettung TTAE Greifer}
 		\label{img:TTGrappleIoU}
 	\end{figure}
 	
	\subsection{Autocrane-Datensatz}
	\label{subsec:AutocraneDatensatz}	
	Im Rahmen der Arbeit wurde ein Datensatz erarbeitet und Projektpartnern zur Verfügung gestellt. Eine spätere Veröffentlichung des Datensatzes unter der Adresse psiori.com ist geplant. Der Datensatz soll zu allgemeinen akademischen und pädagogische Zwecken genutzt werden dürfen. Alternativ kann Zugang zu dem Datensatz über die E-Mail-Adresse info@psiori.com angefragt werden.

	\subsection{Multi-Task}
	\label{subsec:MehrfacheAufgaben}
	Multi-Task-Lernen beschränkt sich nicht auf zwei Aufgaben. Die gezeigte Lösung kann um weitere Aufgaben erweitert werden. In Abbildung \ref{img:AusblickMultiTaskAnsatz} ist der erweitere Ansatz schematisch abgebildet. Wie in den bisherigen Ansätzen werden ausgehend von der Code-Schicht weitere Aufgaben bearbeitet. Durch die weiteren Aufgaben wird die Gewichtung der einzelnen Aufgaben noch schwerer. In Experiment \ref{subsec:AutoMLExperiment}  wurde gezeigt, dass AutoML bei der Gewichtung helfen kann. Dieser erweiterte Ansatz soll in einem Anschlussprojekt für die Praxistauglichkeit untersucht werden. Der Aufwand, das bisher entwickelte Python-Modul zu erweitern, wird sich dabei in Grenzen halten. Insbesondere müssen im Konstruktor Erweiterungen zum Erstellen der Modelle der weiteren Aufgaben getätigt werden. An anderen Stellen wie z. B. der Methode zum Trainieren des Gesamtmodells wird keine Erweiterung notwendig sein, da bereits mit Listen gearbeitet wird.
	\begin{figure}[h]
		\centering
		\includegraphics[width=0.6\textwidth, center]{bilder/FazitUndAusblick/MultiTaskAnsatz.PNG}
		\caption[Ausblick Multi-Task-Ansatz]{Multi-Task-Ansatz}
		\label{img:AusblickMultiTaskAnsatz}
	\end{figure}

	\subsection{Flexibilität der Werkzeuge}
	\label{subsec:FlexibilitätDerWerkzeuge}
	Die bisherigen Module sind starr bei der Anwendung. Es ist notwendig, ein Multi-Aufgaben-Modell zu erstellen, um anschließend den modellbasierten Transfer durchzuführen. Soll ein neuer modellbasierter Transfer durchgeführt werden, ist es zwingend erforderlich dies ausgehend des Multi-Aufgaben-Modells durchzuführen. Es ist nicht möglich einen modellbasierten Transfer, ausgehend von einem vorherigen Transfer durchzuführen. 
	Als mögliche Erweiterung der Module ist es empfehlenswert, eine Flexibilisierung des Gesamtansatzes durchzuführen. Durch eine Anpassung bei der Modellerstellung der Module, ist es möglich, dass die Abfolge nicht mehr notwendig ist. Es ist vorstellbar, dass ein neues Modell ausgehend von einer Architekturdefinition komplett neu trainiert wird, oder ausgehend von einem Autoencoder-Modell mit null bis beliebig vielen weiteren Aufgaben trainiert werden kann. Konkret muss im Konstruktor der Module eine Zusammenführung der bisher getrennten Ansätze durchgeführt werden.     

	\section{Kritische Reflexion}
	\label{sec:KritischeReflexion}

	andere Domäne testen Greifer ist sehr klar	Daten	