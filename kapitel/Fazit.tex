\chapter{Fazit}
\label{chap:Fazit}

	\section{Zusammenfassung}
	\label{sec:Zusammenfassung}
			
	\section{Kritische Reflexion}
	\label{sec:KritischeReflexion}
				
	\section{Ausblick und weitere Arbeiten}
	\label{sec:AusblickWeitereArbeiten}


	\subsection{Transfer auf Greiferdatensatz}
	\label{subsec:TransferGreiferDatensatz}
 	Die Annotationskosten sind je nach Art von Annotation und Aufwand unterschiedlich. Eine einfache Zuordnung eines Bildes zu einer von zwei Klassen liegt dabei im einstelligen Cent-Bereich. Die Markierung eines Objektes in einem Bild mittels Rahmen verursacht Kosten im zweistelligen Cent-Bereich und die Markierung von Winkeln in einem Bild kann mehr als einen Euro kosten. 		

	\todo{Ergebnisse Transfer Greifer}
	
	\subsection{Autocrane-Datensatz}
	\label{subsec:AutocraneDatensatz}	
	Im Rahmen der Arbeit wurde ein Datensatz erarbeitet und Projektpartner zur Verfügung gestellt. Eine spätere Veröffentlichung des Datensatzes unter der Adresse psiori.com ist geplant. Der Datensatz soll zur allgemeinen akademischen und pädagogische Zwecke genutzt werden dürfen. Alternativ kann Zugang zu dem Datensatz über die E-Mail-Adresse info@psiori.com erfragt werden.

	\subsection{Mehrfach-Aufgaben}
	\label{subsec:MehrfacheAufgaben}
	Mehrfach-Aufgaben-Lernen beschränkt sich nicht auf zwei Aufgaben. Die gezeigte Lösung kann um weitere Aufgaben erweitert werden. In Abbildung \ref{img:AusblickMultiTaskAnsatz} ist der erweitere Ansatz Schematisch abgebildet. Wie in den bisherigen Ansätzen werden ausgehend von der Code-Schicht weitere Aufgaben bearbeitet. Durch die weiteren Aufgaben wird die Gewichtung der einzelnen Aufgaben noch schwerer. In Experiment \ref{sec:AutoTransferSecondCriterionAutoenocder} \todo{ref prüfen} wurde gezeigt, dass BOHB bei der Gewichtung helfen kann. Dieser erweiterte Ansatz soll in einem Anschlussprojekt für die Praxistauglichkeit untersucht werden. Der Aufwand das bisher entwickelte Python-Module zu erweitern wird sich dabei in Grenzen halten. Insbesondere müssen im Konstruktor Erweiterungen zum Erstellen der Modelle der weiteren Aufgaben getätigt werden. An anderen Stellen wie z.B. die Methode zum Trainieren des Gesamtmodells wird keine Erweiterung notwendig sein. Es wird schon mit Listen gearbeitet.   
	\begin{figure}[h]
		\centering
		\includegraphics[width=0.5\textwidth, center]{bilder/FazitUndAusblick/MultiTaskAnsatz.PNG}
		\caption[Ausblick Multi-Task-Ansatz]{Multi-Task-Ansatz}
		\label{img:AusblickMultiTaskAnsatz}
	\end{figure}


	\subsection{Flexibilität der Werkzeuge}
	\label{subsec:FlexibilitätDerWerkzeuge}
	Die bisherigen Module sind starr bei der Anwendung. Es ist notwendig ein Multi-Aufgaben-Modell zu erstellen um anschließend den modellbasierten Transfer durchzuführen. Soll ein neuer modellbasierter Transfer durchgeführt werden, ist es zwingend erforderlich dies ausgehend des Multi-Aufgaben-Modells durchzuführen. Es ist  nicht möglich einen modellbasierten Transfer ausgehend von einem vorherigen Transfer durchzuführen. 
	Als mögliche Erweiterung der Module ist es empfehlenswert eine Flexibilisierung des Gesamtansatzes durchzuführen. Durch eine Anpassung bei der Modellerstellung der Module ist es möglich das die Abfolge nicht mehr notwendig ist. Es ist vorstellbar, dass ein neues Modell ausgehend von einer Architekturdefinition komplett neu trainiert wird, oder ausgehend von einem Autoencoder-Modell mit null bis beliebig vielen weiteren Aufgaben trainiert werden kann. Konkret muss im Konstruktor der Module eine Zusammenführung der bisher getrennten Ansätze durchgeführt werden.     
