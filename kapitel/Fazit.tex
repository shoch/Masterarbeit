\chapter{Fazit}
\label{chap:Fazit}

	\section{Zusammenfassung}
	\label{sec:Zusammenfassung}
			
	\section{Kritische Reflexion}
	\label{sec:KritischeReflexion}
				
	\section{Ausblick und weitere Arbeiten}
	\label{sec:AusblickWeitereArbeiten}


	\subsection{Transfer auf Greiferdatensatz}
	\label{subsec:TransferGreiferDatensatz}
 	Die Annotationskosten sind je nach Art von Annotation und Aufwand unterschiedlich. Eine einfache Zuordnung eines Bildes zu einer von zwei Klassen liegt dabei im einstelligen Cent-Bereich. Die Markierung eines Objektes in einem Bild mittels Rahmen verursacht Kosten im zweistelligen Cent-Bereich und die Markierung von Winkeln in einem Bild kann mehr als einen Euro kosten. 		

	\subsection{Autocrane-Datensatz}
	\label{subsec:AutocraneDatensatz}	
	Im Rahmen der Arbeit wurden \~ 12.000 Bilder mit der Annotation "befinden sich Baumstämme im Greifer oder nicht?" und 4.500 Bilder mit der Annotation "Rahmen um den Greifer" genutzt. Diese Beiden Datensätze sind unter der Adresse \todo{Adresse-Datensätze} unter einer \todo{Lizenz} veröffentlicht und können entsprechend der Lizenz genutzt werden.  


	\subsection{Mehrfache Aufgaben}
	\label{subsec:MehrfacheAufgaben}
	

	\subsection{Flexibilität der Werkzeuge}
	\label{subsec:FlexibilitätDerWerkzeuge}
