\listoftodos

\chapter{Einleitung}
\label{chap:Einleitung}
	\todo{Einleitung schreiben}
	Das Finden geeigneter Datenrepräsentationen ist ein bekanntes Problem im Feld der DataScience. Dabei ist bekannt, dass Datenrepräsentation maßgeblich für die Leistungsfähigkeit von maschinellem Lernen verantwortlich ist. Insbesondere hochdimensionale Daten, wie Bilder, haben mit dem Fluch der Dimensionalität[] zu kämpfen. Der Einsatz von Autoencoder[] erlaubt es komprimierte Datenrepräsentationen für Bilder zu finden. Sind dabei ein unüberwachtes Lernverfahren, brauchen also keine beschrifteten Daten.
	In vielen Anwendungsfällen ist es teuer oder schwierig beschriftete Daten für neue Anwendungsfälle zu beschaffen. Transferlernen adressiert diese Problematik. Es hat das Ziel, gute Modelle in einer neuen Domäne basierend von Wissen in einer anderen Domäne zu erstellen.
	
	Die PSIORI GmbH \cite{PSIORIGmbH.2020} ist ein Unternehmen, welches Projekte im Bereich der künstlichen Intelligenz für Kunden durchführt. PSIORI kann dabei auf einen mehrjährigen Erfahrungsschatz im Bereich des maschinellem Lernen zurückgreifen und setzt dabei häufig Autoencoder und das Transferlernen zum Lösen von Aufgabenstellungen ein. Oft ist der Ansatz des Transferlernenes kostengünstiger um ein neues Modell in einer bestehenden Domäne zu erstellen. 

 	(AutoMl auch noch Motivieren)

	\section{Zielsetzung}
	\label{sec:Zielsetzung}
	
	Viele ähnliche Tasks
		- Labeling ist teuer
		- Transfer auf anderes -> reduce personenen aufwände / Kosten
		- Repäsenstaio nauf Domäne zwingens
		
	
		\todo{Zielsetzung schreiben}
	Ziel dieser Arbeit ist es, Werkzeuge zur Kombination der Ansatzes des Autoencoders und des Transferlernenes zu erstellen und an Hand von einem echten Datensatz zu evaluieren.
	\paragraph{SecondCriterionAutoenocder} Der SecondCriterionAutoenocder (SCAE) ist ein Werkzeug welches beim Erstellen eines Autoencoder mit weiterer Verlustfunktion unterstützt.  
    \paragraph{TransferSecondCriterionAutoenocder} Der TransferSecondCriterionAutoenocder (TSCAE) ersetzt die Verlustfunktion eines SCAE.
 	\paragraph{AutoTransferSecondCriterionAutoenocder} Der AutoTransferSecondCriterionAutoenocder (AutoTSCAE) erweitert den TSCAE um eine automatische Hyperparametersuche. 
  		
	\section{Vorgehen}
	\label{sec:Vorgehen}
			\todo{Vorgehen schreiben, CRISP-DM + Werkezugerstellung}
	Werkezuge erstellen Mnsit + Emnist
	 1.
	 2.
	 
	Modelle Iterativ verbessern
		1. fit
		2. fitgenerator
		3. 80.000	
	3. AutoMl
		
	
	
	[MNIST] [EMNIST]
	



