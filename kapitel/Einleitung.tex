\listoftodos

\chapter{Einleitung}
\label{chap:Einleitung}
	Die PSIORI GmbH \cite{PSIORIGmbH.2020} ist ein Projekt- und Beratungsunternehmen im Bereich der Digitalisierung und Data Science. In vielen Digitalisierungs- und Automatisierungsprojekten fallen eine Vielzahl von Aufgaben an, welche mittels künstlicher Intelligenz gelöst werden können. In der Regel wird dabei für jede Aufgabe, welche mittels künstlicher Intelligenz gelöst werden soll, eine aufgabenspezifische Annotation benötigt. Das Annotieren von Daten ist teuer. Zum Beispiel sollten in einem Projekt circa 80.000 Bilder mittels zehn verschiedenen Annotationen beschriftet werden, was zu Kosten von circa 240.000 Euro führt. Zusätzliche Kosten entstehen durch Personenaufwände beim Erstellen ähnlicher Modelle für Aufgaben in einer Domäne. Die Kosten solcher Data Science Projekte können gesenkt werden, wenn weniger annotierte Daten genutzt werden müssen.
	
	Ziel dieser Arbeit ist es, herauszufinden, ob es, von Datenrepräsentationen einer Domäne ausgehend, möglich ist, den Einsatz von annotierten Daten zu reduzieren. Im Idealfall ist es möglich, ausgehend von einer geeigneten Repräsentation, schnell und robust neue Aufgaben in der Domäne bearbeiten zu können. Die eingesetzten Techniken sollen dabei nachhaltig bereitgestellt werden und von anderen Data Scientisten eingesetzt werden können. 
	
	Das Vorgehen des praktischen Teils der Arbeit orientiert sich an dem CRISP-DM Model \cite{Shearer.2000}, erfolgt also iterativ. Um die Ergebnisse nachhaltig anderen Entwicklern zur Verfügung zu stellen, wird vor die Modellierungsphase eine Phase zur Werkzeugerstellung eingefügt. Die Werkzeuge unterstützen einen Data Scientisten bei der Anwendung der vorgeschlagenen Techniken. Die durchgeführten Experimente sollen insbesondere zeigen, dass die Werkzeuge und die dahinterstehenden Techniken funktionieren. 
	
	Dieses Dokument gliedert sich in ein Grundlagenkapitel, in welchem die notwendigen theoretischen Grundlagen, das Umfeld sowie die genutzten Datensätze der Arbeit vorgestellt werden. Der Hauptteil der Arbeit beinhaltet eine  Vorstellung und Evaluierung eines Multi-Task-Ansatzes. Auf diesem Ansatz aufbauend, wird ein Transfer-Learning-Ansatz mit einer Erweiterung des automatischen maschinellen Lernens sowohl vorgestellt als auch einer Evaluierung unterzogen. Abgeschlossen wird die Arbeit mit einer Bewertung der Lösung der Problemstellung und einem umfassenden Ausblick auf mögliche Erweiterungen und Potentiale.
