\listoftodos

\chapter{Einleitung}
\label{chap:Einleitung}
	Die PSIORI GmbH \cite{PSIORIGmbH.2020} ist ein Projekt- und Beratungsunternehmen im Bereich der Digitalisierung und Data Science. In vielen Digitalisierungs- und Automatisierungsprojekten fallen eine Vielzahl von Aufgaben an, welche mittels künstlicher Intelligenz gelöst werden können. In der Regel wird dabei für jede Aufgabe, welche mittels künstlicher Intelligenz gelöst werden soll, eine eigene Annotation benötigt. Das Annotieren von Daten kann sehr schnell sehr teuer werden. Z. B. sollten in einem Projekt circa 80.000 Bilder mittels zehn verschiedenen Annotationen beschriftet werden, was zu Kosten von circa 240.000 Euro führt. Interessanterweise sind die Kosten je nach Annotation und pro Bild sehr unterschiedlich. Die Einordnung eines Bildes ob es zu einer von zwei Klassen gehört liegt dabei im einstelligen Cent-Bereich. Die Markierung eines Objektes in einem Bild mittels Rahmen kostet im zweistelligen Cent-Bereich und die Markierung von Winkeln in einem Bild kann mehr als einen Euro Kosten. Zusätzliche Kosten entstehen durch Personenaufwände beim erstellen ähnlicher Modelle für Aufgaben in einer Domäne. Die Kosten solcher Data Science Projekte können also gesengt werden wenn weniger Annotierte Daten, insbesondere der teuren Annotationen genutzt werden müssen und wenn weniger Personenaufwände entstehen.
	
	Ziel dieser Arbeit ist es, herauszufinden, ob es möglich ist, nur einen Teil der Daten zu beschriften, insbesondere ob durch den Einsatz der günstigeren Labels, teurere Labels gespart werden können. Jeder Transfer von Informationen von den günstigeren zu den teuren Labels beinhaltet einen Aufgabenwechsel. Dieser Aufgabenwechsel soll dabei nach Möglichkeit schnell und robust stattfinden können. Die eingesetzten Techniken sollen dabei nachhaltig bereitgestellt werden und von anderen Data Scientisten eingesetzt werden können. 
	
	Das Vorgehen des praktischen Teils der Arbeit orientiert sich an dem CRISP-DM Model \cite{Shearer.2000} erfolgt also iterativ. Um die Ergebnisse nachhaltig anderen Entwicklern zur Verfügung zu stellen, wird vor die Modellierungsphase eine Phase zur Werkzeugerstellung eingefügt. Die durchgeführten Experimente sollen insbesondere zeigen, dass die Werkzeuge und die dahinterstehende Idee funktionieren. Der ganze Prozess wird dabei dreimal durchgeführt. In der ersten Prozessiteration soll ein Werkzeug zum Finden von geeigneten Repräsentationen erstellt werden. In der zweiten Prozessiteration soll ein Werkzeug zum wechseln von Aufgaben erstellen, ausgehend von dem Modell der ersten Iteration erstellt werden. In der dritten Prozessiteration wird das Werkzeug der zweiten Prozessiteration automatisiert.
	
	Der theoretische Teil der Arbeit gliedert sich in ein Grundlagenkapitel in welchem die notwendigen theoretischen Grundlagen und das Umfeld sowie die Daten der Arbeit dargestellt werden. Ein Kapitel, in dem die Werkzeuge im Detail vorgestellt werden, gefolgt von einem Kapitel, in dem die durchgeführten Experimente und ihre Ergebnisse dargestellt werden. Das abschließende Kapitel enthält eine kurze Zusammenfassung der Arbeit und gibt einen Ausblick auf mögliche Verbesserungen und Erweiterungen der Werkzeuge.
