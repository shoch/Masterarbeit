\chapter{Einleitung}
\label{chap:Einleitung}
	Die PSIORI GmbH \cite{PSIORIGmbH.2020} ist ein Projekt- und Beratungsunternehmen im Bereich der Digitalisierung und Data Science. In vielen Digitalisierungs- und Automatisierungsprojekten fallen eine Vielzahl von Aufgaben an, die mittels künstlicher Intelligenz gelöst werden können. In der Regel wird dabei für jede Aufgabe, die mittels künstlicher Intelligenz gelöst werden soll, eine aufgabenspezifische Annotation benötigt. Das Annotieren von Daten ist teuer. Bei einem Beispielhaften Projekt bei PSIORI, bei dem 80.000 Bilder durch zehn Annotationen beschriftet werden sollten, entstehen Kosten von rund 240.000 Euro. Zusätzliche Kosten entstehen durch Personalaufwände beim Erstellen ähnlicher Modelle für Aufgaben in einer Domäne. Die Kosten von Data Science Projekte können jedoch gesenkt werden, wenn weniger annotierte Daten genutzt werden müssen.
	
	Ziel dieser Arbeit ist es ein Verfahren zu entwickeln um den Einsatz von annotierten Daten zu reduzieren, ausgehend von Datenrepräsentationen einer Domäne. Im Idealfall ist es möglich, ausgehend von einer geeigneten Repräsentation, schnell und robust neue Aufgaben in der Domäne zu bearbeiten. Die eingesetzten Techniken sollen nachhaltig bereitgestellt und von weiteren Data Scientisten eingesetzt werden können.
	
	Das Vorgehen des praktischen Teils der Arbeit orientiert sich an dem CRISP-DM Model \cite{Shearer.2000}, erfolgt also iterativ. Um die Ergebnisse nachhaltig anderen Entwicklern zur Verfügung zu stellen, wird vor die Modellierungsphase eine Phase zur Werkzeugerstellung eingefügt. Die Werkzeuge unterstützen Data Scientisten bei der Anwendung der vorgeschlagenen Techniken. Die durchgeführten Experimente sollen insbesondere zeigen, dass die Werkzeuge und die dahinterstehenden Techniken funktionieren. 
	
	Dieses Dokument gliedert sich in ein Grundlagenkapitel, in dem die notwendigen theoretischen Grundlagen, das Umfeld sowie die genutzten Datensätze der Arbeit vorgestellt werden. Das Kapitel Experimente und Werkzeuge beinhaltet eine Vorstellung und Evaluierung eines Multi-Task-Lernen-Ansatzes. Auf diesem Ansatz aufbauend, wird ein Transferlernen-Ansatz mit einer Erweiterung des automatischen maschinellen Lernens sowohl vorgestellt als auch einer Evaluierung unterzogen. Abgeschlossen wird die Arbeit mit einem Fazit das einer Bewertung der Lösung der Problemstellung enthält und einem umfassenden Ausblick auf mögliche Erweiterungen und Potentiale.
